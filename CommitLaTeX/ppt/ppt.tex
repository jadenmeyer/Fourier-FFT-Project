\documentclass{beamer}
\usepackage{graphicx}
%Information to be included in the title page:
\title{FFT Algorithim Comparison for GPUs and CPUs}
\author{Jaden Meyer and Jayden Burr}

\begin{document}

\frame{\titlepage}

\begin{frame}
\frametitle{FFT Algorithim}
This is some text in the first frame. This is some text in the first frame. This is some text in the first frame.
\end{frame}

\begin{frame}
\frametitle{GPU Speeds}
The FFT Algorithim has time complexity of $O(NlogN)$. The average GPU has hundreds to tens of thousands of cores. It can compute things in parallel.\\
\vspace{1mm}
Using an average graphics card we expect 4-8 TFLOPs ($10^{12}$ operations):\\
\vspace{1mm}
\pause
For a DFT with $N = 1 million$: $ (10^6)^2 \cdot \frac{1}{4\cdot10^{12}} = 0.25$ seconds\\
\pause
For a FFT with $N= 1 million$: $ (10^6) \cdot log_2(10^6) \cdot \frac{1}{4\cdot10^{12}} = 0.000005$ seconds\\
\pause
\vspace{1mm}
That is FAST...

\end{frame}

\begin{frame}
    \frametitle{CPU Speeds}
    A CPU has only a handful of cores. It does not use parallelization and is very weak at 'indirect' computations. Occasionally the core will use 1 or 2 threads.\\
    \vspace{1mm}
    \pause
    Most modern CPUs use around 4-5GHz which means a single-core operation speed of: 0.02 TFLOPs:\\
    For a DFT with $N = 1 million$: $ (10^6)^2 \cdot \frac{1}{0.02\cdot10^{12}} = 50$ seconds\\
    \pause
    For a FFT with $N= 1 million$: $ (10^6) \cdot log_2(10^6) \cdot \frac{1}{0.02\cdot10^{12}} = 0.001$ seconds\\
    \pause
    \vspace{1mm}
    That is quite slow...
\end{frame}

\begin{frame}
    \frametitle{Results?}
    
\end{frame}

\end{document}